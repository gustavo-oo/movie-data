Este trabalho teve como objetivo aplicar técnicas de mineração de dados para explorar padrões e tendências na indústria do cinema, abrangendo desde a coleta e tratamento dos dados até a análise exploratória e a construção de modelos preditivos. Para isso, foi utilizada a \acrshort{API} do \acrfull{TMDB} como fonte primária de dados, garantindo um amplo espectro de informações sobre filmes lançados entre 2013 e 2023.

A primeira etapa do estudo consistiu na coleta e seleção dos dados, seguida por um processo detalhado de pré-processamento e transformação, onde identificadores foram mapeados para seus valores correspondentes, colunas irrelevantes foram removidas e filtros foram aplicados para garantir a qualidade dos dados utilizados. Essa etapa foi crucial para assegurar que as análises subsequentes fossem realizadas sobre um conjunto de dados limpo e estruturado.

A análise exploratória revelou diversas características da indústria cinematográfica ao longo dos anos. A predominância dos Estados Unidos na produção cinematográfica global ficou evidente, destacando a forte influência da indústria hollywoodiana. Além disso, os gêneros drama e comédia lideraram em popularidade tanto globalmente quanto nos mercados do Brasil e dos Estados Unidos. Viu-se também uma queda nas produções mundias no ano de 2020, provavelmente atribuída a pandemia do COVID-19.

Outro aspecto relevante foi a análise da representatividade de gênero no elenco e na equipe de produção, tanto no mundo quanto no Brasil e EUA, que demonstrou um desequilíbrio persistente. Observou-se que os homens continuam a ocupar a maioria dos papéis principais e cargos de produção, com apenas algumas setores com maior presença feminina. Percebeu-se, em conjunto com as demais análises, que não houve avanços significativos na igualdade entre os gêneros ao longo dos anos analisados.

No âmbito econômico, foram analisadas diversas relações entre orçamento, receita e retorno sobre investimento. Constatou-se que os gêneros com maior orçamento mediano são \quotes{animação}, \quotes{família} e \quotes{aventura}, enquanto os gêneros com os maiores retornos sobre investimento (\acrshort{ROI}) são \quotes{família}, \quotes{animação}, \quotes{aventura}, \quotes{ação}, \quotes{ficção científica} e \quotes{comédia}. Além disso, observou-se que filmes pertencentes a coleções possuem um desempenho financeiro superior em relação a filmes individuais, reforçando a importância do fator franquia para a viabilidade financeira de uma produção. A análise econômica também destacou diferenças significativas entre os mercados globais, com a China apresentando o maior orçamento mediano de filmes e um dos maiores \acrshort{ROI}s.

A modelagem preditiva buscou estimar o sucesso financeiro de um filme antes de seu lançamento, utilizando variáveis como orçamento, presença em coleção e gênero. Durante essa etapa, observou-se que o modelo de regressão linear apresentou limitações significativas, sendo incapaz de capturar adequadamente a complexidade dos fatores que influenciam a receita dos filmes. Em contrapartida, a abordagem baseada em árvores de decisão demonstrou resultados mais consistentes, fornecendo previsões mais precisas e interpretáveis sobre o desempenho financeiro das produções. Os modelos indicaram que esses fatores são determinantes para prever o desempenho comercial de uma produção.

Dentre as limitações do estudo, destaca-se a dependência da base de dados do \textit{\acrshort{TMDB}}, que pode apresentar lacunas ou inconsistências em determinadas informações. Além disso, a análise não considerou aspectos subjetivos como qualidade do roteiro, impacto cultural e campanhas de marketing, que podem ter influência significativa no sucesso de um filme. Outro ponto adicional foi a ausência de uma seleção mais cuidadosa das variáveis de entrada dos modelos preditivos, o que provavelmente impactou o resultado obtido.

Como proposta para trabalhos futuros, sugere-se a ampliação da base de dados para incluir outras fontes de informações, na busca por dados mais robustos de filmes brasileiros. Além disso, a aplicação de algoritmos não lineares e uma melhor seleção de atributos, poderia fornecer previsões mais precisas sobre o sucesso financeiro de um filme. Por fim, o desenvolvimento de ferramentas visuais interativas permitiria um acompanhamento constante dessas análises, permitindo observar as evoluções nas dinâmicas de gênero, conteúdo de filmes e países produtores.
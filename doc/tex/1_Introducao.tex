
%\url{http://www.escritacientifica.com/}

%%%%%%%%%%%%%%%%%%%%%%%%%%%%%%%%%%%%%%%%%%%%%%%%%%%%%%%%%%%%%%%%%%%%%%%%%%%%%%%%
%%%%%%%%%%%%%%%%%%%%%%%%%%%%%%%%%%%%%%%%%%%%%%%%%%%%%%%%%%%%%%%%%%%%%%%%%%%%%%%%
%%%%%%%%%%%%%%%%%%%%%%%%%%%%%%%%%%%%%%%%%%%%%%%%%%%%%%%%%%%%%%%%%%%%%%%%%%%%%%%%

\section{Problema}%
O mercado cinematográfico é um setor multifacetado, influenciado por diferentes fatores. No entanto, a análise aprofundada desses aspectos é frequentemente limitada por abordagens tradicionais, que não exploram de maneira eficiente os grandes volumes de dados disponíveis. 

Além disso, a comparação entre diferentes indústrias, como as do Brasil e dos Estados Unidos, carece de estudos que integrem múltiplas variáveis para identificar padrões, tendências e disparidades entre os mercados, isso somado a falta de uma abordagem sistemática baseada em mineração de dados impede uma compreensão detalhada de como aspectos econômicos, culturais e de gênero impactam a produção e o desempenho dos filmes dos dois países.

Dessa forma, busca-se aplicar técnicas analíticas para estruturar e interpretar essas informações, permitindo uma avaliação mais precisa da indústria cinematográfica e suas particularidades.

\section{Motivação}
A aplicação de técnicas de mineração de dados pode revelar informações relevantes, proporcionando uma visão mais objetiva e estruturada da indústria cinematográfica. Esse processo permite evidenciar discrepâncias entre diferentes países, oferecendo um panorama detalhado do estado atual do setor. Ao integrar múltiplas variáveis, a análise de dados possibilita identificar padrões e tendências que não seriam perceptíveis por abordagens tradicionais, contribuindo para uma compreensão mais ampla do funcionamento e das particularidades do setor.

\section{Objetivo}
O objetivo deste trabalho é aplicar técnicas de mineração de dados para analisar diferentes aspectos da indústria cinematográfica, integrando análises econômicas, culturais e de gênero. Além disso, busca-se realizar comparações específicas entre as indústrias do Brasil e dos EUA, destacando semelhanças e particularidades nesses dois mercados.

\section{Objetivos Específicos}
De forma a atender ao objetivo central do trabalho, os seguintes objetivos específicos foram definidos:
\begin{itemize}
    \item Investigar o impacto de aspectos quantitativos e qualitativos na receita e na performance econômica dos filmes, com destaque para particularidades nos mercados brasileiro e estadunidense;
    \item Examinar padrões de diversidade de gênero no elenco e na produção, incluindo análises comparativas entre Brasil e EUA;
    \item Analisar os temas e gêneros predominantes nas produções cinematográficas e sua evolução ao longo dos anos, avaliando tendências específicas de cada país;
    \item Criar visualizações, como mapas coropléticos, para destacar a distribuição geográfica das produções e as diferenças entre regiões;
    \item Aplicar algoritmos de mineração de dados, como regressão e árvore de decisão, para prever o sucesso financeiro dos filmes, utilizando variáveis que sejam relevantes em um contexto global e para as indústrias analisadas.
\end{itemize}


\section{Metodologia}
A metodologia deste trabalho foi desenvolvida com base em etapas práticas voltadas para a análise de dados cinematográficos, conforme detalhado abaixo:
\begin{itemize}
    \item Pesquisa inicial: Identificação de aspectos principais dos filmes e informações relevantes a serem analisadas, com a definição do intervalo de tempo a ser estudado;
    \item Busca de fontes de dados: Avaliação de fontes públicas disponíveis com atributos suficientes para o atendimento dos objetivos;
    \item Coleta de dados: Extração automática dos dados utilizando \textit{scripts} em Python, selecionando campos necessários para as análises
    \item Preparação dos dados: Avaliação, formatação e filtragem dos dados obtidos para garantir qualidade e consistência na análise;
    \item Criação de visualizações: Desenvolvimento de gráficos e mapas utilizando \textit{scripts} em Python para permitir observações mais profundas sobre a distribuição dos dados e os seus relacionamentos;
    \item Modelagem preditiva: Implementação de algoritmos de regressão para prever lucro e sucesso financeiro dos filmes, bem como entender a contribuição das características de um filme nesse resultado;
\end{itemize}

\section{Organização dos Capítulos}
O trabalho está estruturado em cinco capítulos principais:

\begin{itemize}
\item Capítulo 1 - Introdução: Contextualiza o problema, apresenta os objetivos e a metodologia do trabalho;

\item Capítulo 2 - Fundamentação Teórica: Explora conceitos fundamentais de análise de dados, trazendo a bagagem necessária para o entendimento dos demais capítulos.

\item Capítulo 3 - Preparação dos Dados: Detalha o processo de coleta, tratamento e transformação dos dados para as análises subsequentes.

\item Capítulo 4 - Análise dos Dados: Apresenta os resultados obtidos a partir das análises econômicas, culturais e de gênero, com base nos métodos descritos.

\item Capítulo 5 - Conclusões: Resume os principais achados, discute as limitações do trabalho e sugere caminhos para estudos futuros.

\end{itemize}
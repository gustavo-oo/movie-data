% A indústria cinematográfica desempenha um papel fundamental na economia e na cultura global, influenciando comportamentos, tendências e gerando receitas bilionárias. Com o avanço da tecnologia e o crescimento exponencial de dados disponíveis, a aplicação de algoritmos de mineração de dados na análise do setor tornou-se uma ferramenta essencial para extrair padrões e tendências significativas. Este trabalho explora a aplicação de técnicas de mineração de dados na indústria do cinema, visando compreender a distribuição de filmes ao longo dos anos, padrões de desempenho financeiro e características de gênero do elenco e da equipe de produção.

% Para isso, foram coletados dados da \acrshort{API} do \acrfull{TMDB}, abrangendo informações de milhares de filmes lançados entre 2013 e 2023. Após a coleta, os dados passaram por um rigoroso processo de seleção, tratamento e transformação, garantindo sua integridade e qualidade.

% As análises realizadas revelaram que os Estados Unidos foram o país com maior número de produções cinematográficas em todos os anos analisados. No que se refere aos gêneros cinematográficos, identificou-se que drama e comédia lideram tanto globalmente quanto nos mercados do Brasil e EUA. A análise de representatividade de gênero no elenco e na equipe de produção indicou que não houve evolução significativa na igualdade entre os gêneros ao longo dos anos, sendo que os homens continuam a ocupar mais papéis principais do que as mulheres.

% Em relação à análise econômica, constatou-se que determinados gêneros, como ação e ficção científica, apresentam os maiores orçamentos médios, enquanto gêneros como terror e documentário demonstram os maiores retornos sobre investimento (\acrshort{ROI}). Além disso, filmes pertencentes a coleções tiveram um \acrshort{ROI} superior em comparação com filmes individuais, reforçando a importância do fator franquia na viabilidade financeira de uma produção.

% Por fim, foram desenvolvidos modelos preditivos para estimar o sucesso financeiro dos filmes com base em variáveis disponíveis antes do lançamento. Os resultados indicaram que a presença em coleções, o gênero do filme e o orçamento são fatores determinantes para a previsão do desempenho comercial. 

% Esses achados reforçam a importância da análise de dados para a tomada de decisões estratégicas na indústria cinematográfica, além de revelar padrões que refletem aspectos culturais e desigualdades de gênero dentro do setor.

A indústria cinematográfica, enquanto fenômeno cultural e econômico global, enfrenta desafios complexos que vão desde a previsão de sucesso financeiro até a promoção de diversidade em suas produções. Este trabalho aplica algoritmos de mineração de dados para desvendar padrões nesse setor, utilizando como base o The Movie Database (TMDB), que reúne informações detalhadas sobre filmes lançados entre 2013 e 2023. Seguindo as etapas do \acrfull{KDD}, foram coletados e processados dados econômicos (orçamento, receita), demográficos (gênero do elenco e equipe) e temáticos (gêneros, palavras-chave). As análises exploratórias revelaram que filmes de drama e comédia predominam globalmente, enquanto produções vinculadas a coleções apresentam maior retorno sobre investimento (\acrshort{ROI}). Aplicou-se os modelos preditivos de regressão linear e árvores de decisão para prever o sucesso financeiro, identificando variáveis como \quotes{pertencimento a coleções} e \quotes{país produtor} como decisivas. Comparações adicionais entre as produções do Brasil e EUA destacaram semelhanças, tanto na menor representatividade de mulheres em cargos principais, quanto em conteúdos de filmes. O estudo demonstra o potencial da mineração de dados para orientar decisões estratégicas, ao mesmo tempo que expõe lacunas persistentes no setor.
The film industry, as a global cultural and economic phenomenon, faces complex challenges ranging from financial success prediction to promoting diversity in its productions. This study applies data mining algorithms to uncover patterns within this sector, using \acrfull{TMDB} as the primary data source, containing detailed information on films released between 2013 and 2023. Following the steps of the \acrfull{KDD} process, economic (budget, revenue), demographic (cast and production team gender), and thematic (genres, keywords) data were collected and processed. Exploratory analyses revealed that drama and comedy films dominate globally, while productions associated with collections exhibit higher return on investment (\acrshort{ROI}). Predictive models, including linear regression and decision trees, were applied to forecast financial success, identifying variables such as \quotes{belongs to collection} and \quotes{production countries} as decisive factors. Additional comparisons between Brazilian and U.S. productions highlighted similarities, particularly in the underrepresentation of women in leading roles and in film content trends. The study demonstrates the potential of data mining to guide strategic decision-making while exposing persistent gaps within the industry.